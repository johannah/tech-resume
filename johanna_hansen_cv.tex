%%%%%%%%%%%%%%%%%%%%%%%%%%%%%%%%%%%%%%%%%
% Medium Length Professional CV
% LaTeX Template
% Version 2.0 (8/5/13)
%
% This template has been downloaded from:
% http://www.LaTeXTemplates.com
%
% Original author:
% Trey Hunner (http://www.treyhunner.com/)
%
% Important note:
% This template requires the resume.cls file to be in the same directory as the
% .tex file. The resume.cls file provides the resume style used for structuring the
% document.
%
%%%%%%%%%%%%%%%%%%%%%%%%%%%%%%%%%%%%%%%%%

%----------------------------------------------------------------------------------------
%	PACKAGES AND OTHER DOCUMENT CONFIGURATIONS
%----------------------------------------------------------------------------------------

\documentclass{resume} % Use the custom resume.cls style

\usepackage[left=0.75in,top=0.6in,right=0.75in,bottom=0.6in]{geometry} % Document margins

\name{Johanna Hansen} % Your name
%\address{Montreal, Quebec} % Your address
\address{ johanna.hansen@mail.mcgill.ca \\ http://johannah.github.io } % Your phone number and email
\begin{document}
	\vspace{-.5cm}
\begin{rSection}{Education}
{\bf McGill University, Montreal, QC} \hfill {\em 2016 -- 2022} \\ 
\setlength{\parindent}{3ex}
\indent{Ph.D. in Computer Science (expected), Mobile Robotics Lab} \\
\indent{Learning Robotic Policies with Physically Consistent World Models}\\
\indent{under the supervision of Dr. Gregory Dudek and Dr. Joelle Pineau}\\
\noindent {\bf University of Texas at San Antonio, San Antonio, TX} \hfill {\em 2012 -- 2015} \\ 
\indent {Graduate coursework (30 hours) in Electrical Engineering, Digital Signal Processing} \\
\noindent {{\bf Texas State University, San Marcos, TX} \hfill {\em 2007 -- 2011}} \\
\indent{B.S. in Electrical Engineering, Networking and Communication} \\
%\indent{Capstone Project on Identifying Devices in the Power Grid by their Current Signatures} \\
\indent{B.S. in Resource and Environmental Geography}
	\vspace{-.5cm}
\end{rSection}
\begin{rSection}{Technical Skills}{}{}{}
\noindent{\textbf{Expertise:} Robotics, Machine Learning, Deep Learning, Reinforcement Learning, Visuotactile Sensing, Oceanographic Sensing, Scientific Machine Learning, Earth and Environmental Science } \\
\noindent{\textbf{Software}: Scientific Python (numpy, pytorch, scikit-learn, pandas), ROS, C, Matlab, AWS} \\
\noindent{\textbf{Hardware}: Custom Sensors, Embedded Systems, Localization Systems, Marine Instrumentation} \\
%\noindent{\textbf{Data}: Images, Geospatial, Time Series, Scientific Sensors, Robot Navigation}
%\item{\emph{Languages}: English (native), Spanish (basic), French (learning)}
	\vspace{-.5cm}
\end{rSection}
\begin{rSection}{Experience}
\begin{rSubsection}{McGill University}{Jan 2016--current}{Graduate Researcher, Mobile Robotics Lab / Mila}{Montreal, QC}
\item{Research on model-based decision making with physics-informed learned models for robotic agents.}
\item{Led research and design of marine sampling campaign with low-cost floating sensors. Agents perform online trajectory planning with physics-based ocean simulation models to learn to distribute floating sensors. Ranked 5 of 44 (top university) in related DARPA Forecasting Floats in Turbulence Challenge.}
\item{Spearheaded collaboration with ecologists to develop automatic classification of zooplankton.}
\item{Built and designed system for portable underwater vehicle localization using low-cost components.} 
%\item{Developed technique for calculating iceberg volume combining data images and sonar data collected from robotic platforms.}
\end{rSubsection}
	\vspace{-.2cm}
	\begin{rSubsection}{Samsung AI Center (SAIC)}{2021-current}{Part-Time Research Intern, Tactile Sensing Group}{Montreal, QC}
\item{Research on multitask learning with contact-based grounding for complex manipulation tasks. }
\item{Developed simulation codebase for visuotactile sensing and reinforcement learning. }
\end{rSubsection}
	\vspace{-.2cm}
\begin{rSubsection}{NASA Jet Propulsion Lab (JPL)}{Summer 2019}{Research Intern and Remote Affiliate, Mobility and Robotics Section}{Pasadena, CA}
\item{Worked on machine vision aspects of the Mars Sample Return Project. Implemented state-of-the-art geometric and direct object localization methods for finding sample tubes on the Martian terrain.}
\item{Assisted in collection of new Mars Sample Return vision dataset with realistic geology.}
%\item{Research on a model which learns to contextually switch between different localization methods to facilitate efficient autonomous rover sample tube collection.}
\end{rSubsection}
	\vspace{-.2cm}
\begin{rSubsection}{Woods Hole Oceanographic Institution (WHOI)}{Jan 2014 -- Sept 2015}{Autonomous Underwater Vehicle Engineer, National Deep Submergence Facility}{Woods Hole, MA}
\item{Software/Data/Electrical Engineer for deep-diving autonomous underwater vehicles (AUVs) working in research and ship-board operational environments. }
\item{Assisted in all aspects of at-sea operations including AUV deployments, hardware repair and maintenance, autonomous navigation, software development, networking and communications for robots and staff, acoustic and visual mapping of the seafloor, and scientific data analysis. Developed automated spatio-temporal processing pipeline for high-resolution multibeam, sidescan sonar, and optical maps.}
%\item{Assisted in overhaul of navigation (GPS/USBL/LBL/DR) processing and visualization.}
%\item{Developed user-interfaces (QT), vehicle control code (C++), data processing code (Python/Matlab), and device drivers primarily in Linux.}
\end{rSubsection}
	\vspace{-.2cm}
	\newpage
\begin{rSubsection}{Southwest Research Institute (SwRI)}{Jan 2012 -- Dec 2013}{Engineer, Automation and Data Systems Division}{San Antonio, TX}{}
%\item{Software/Electrical Applied Research Engineer for research, commercial, and government clients.}
\item{Primary end-to-end software engineer building a mapping sensor consisting of acoustic transducers, DSP, camera, and embedded computer with remote control and interpretation. Developed sampling, filtering, visualization scheme and beamforming calibration routine for live acoustic data. Secret Clearance.}
\end{rSubsection}
\begin{rSubsection}{Lower Colorado River Authority (LCRA)}{Jan 2011 -- Dec 2011}{Engineering Coop, Telecommunications Department}{Austin, TX}
\item{Designed and configured SONET, optical fiber, Ethernet, and microwave systems for critical communication infrastructure including power generation/distribution, dam and irrigation control, and emergency response coordination. }
%\item{Project lead for pilot irrigation control system using networked 900 MHz Radios.}
\end{rSubsection}
\end{rSection}
	\vspace{-.5cm}
\begin{rSection}{Academic Papers}

(\emph{Working Draft}) \textit{Transfer Learning with Differentiable Physics-Informed Priors}

(\emph{Working Draft}) \textit{Contact Sketching for Learning Multitask Manipulation Policies}

(\emph{Working Draft}) \textit{Planning with Learned State Associations }

	\textbf{Hansen, J.*}, Kastner, K.*, Huang, Y., Courville, A., Meger, D., Dudek, G., \textit{Learning to Manipulate from Pixels on Rigid Body Robots with a Kinematic Critic}, (under review), 2022

\textbf{Hansen, J.}, Hogan, F., Rivkin, D., Meger, D., Jenkin, M., Dudek, G., \textit{Visuotactile-RL: Learning Multimodal Manipulation Policies with Deep Reinforcement Learning}, ICRA, 2022


Huang, Y.*, Yao, Y.*, \textbf{Hansen, J.*}, Mallette, J., Manjanna, S., Dudek, G.,  Meger, D., \textit{An Autonomous Probing System for Collecting Measurements at Depth from Small Surface Vehicles}, MTS/IEEE OCEANS, 2021, \emph{{(Top 20 Student Submission)}}.

Pham, T., Seto, W., Daftry, S., Ridge, B., \textbf{Hansen, J.}, Thrush, T., Van der Merwe, M., Maggiolino, G., Brinkman, A., Mayo, J., Cheng, Y., Padgett, C., Kulczycki, E., Detry, R., \textit{Rover Relocalization for Mars Sample Return by Virtual Template Synthesis and Matching},  IEEE Robotics and Automation Letters, 2021.

\textbf{Hansen, J.}, Manjanna, S.,  Quattrini, L. A., Rekleitis, I., Dudek, G., \textit{Autonomous Marine Sampling Enhanced by Strategically Deployed Drifters in Marine Flow Fields}, MTS/IEEE OCEANS, 2018, \emph{{(Top 20 Student Submission)}}.

\textbf{Hansen, J.}, Dudek, G., \textit{Coverage Optimization with Non-Actuated, Floating Mobile Sensors using Iterative Trajectory Planning in Marine Flow Fields}, IEEE International Conference on Intelligent Robots (IROS), 2018.

\textbf{Hansen, J.}*, Kastner, K.*, Courville, A., Dudek, G.,  \textit{Planning in Dynamic Environments with Conditional Autoregressive Models}, International Conference on Machine Learning (ICML), workshop on Prediction and Generative Modeling in Reinforcement Learning, 2018.

Henderson P., Chang, W.D., Shkurti, F., \textbf{Hansen, J.}, Meger, D., Dudek G., 
\textit{Benchmark Environments for Multitask Learning in Continuous Domains}, International Conference on Machine Learning (ICML), workshop on Lifelong Learning, 2018, https://arxiv.org/abs/1708.04352. 

Manjanna, S., \textbf{ Hansen, J. }, Quattrini, L. A., Rekleitis, I., Dudek, G., 
\textit{Collaborative Sampling Using Heterogeneous Marine Robots Driven by Visual Cues}, Canadian Conference on Computer and Robot Vision (CRV), 2017. 
                
Quattrini L. A., Rekleitis, I., Manjanna, S., Kakodkar, N., \textbf{Hansen, J.},   Dudek, G.,  Bobadilla, L.,  Anderson, J., and Smith, R.,
            \textit{Data Correlation and Comparison from Multiple Sensors over a Coral Reef with a Team of Heterogeneous Aquatic Robots},
             International Symposium on Experimental Robotics (ISER),
              2016.

\textbf{Hansen, J.}, Fourie, D., Kinsey, J., Pontbriand, C., Ware, J., Farr, N., Kaiser, C., and Tivey, M., \textit{Autonomous Acoustic-Aided Optical Localization for Data Transfer}, MTS/IEEE OCEANS, 2015.

Pontbriand, C., Farr, N., Fourie, D., \textbf{Hansen, J.},  Kinsey, J., Pelletier, J., and Ware, J., 
 \textit{Wireless Data Harvesting Using the AUV Sentry and WHOI Optical Modem}, MTS/IEEE OCEANS, 2015.

\textbf{Hansen, J.}, Wilden, G., Abbott, B., and Green, R., \textit{The Ultrasonic Culvert
Inspection System (UCIS): A Low-Cost Device for Conduit Inspection}, 2014 Transportation
Research Board 93rd Annual Meeting. 
\end{rSection}
\vspace{-.2cm}
\begin{rSection}{Professional Activities}
\begin{rSubsection}{Invited Presentations}{}{}{}
\item{2019: Tutorial on Model-Based Reinforcement Learning at AI4Good Summer School}
\item{2019: GRIL Presentation on Robotic Sampling in Aquatic Environments}	
\item{2018: PyLadies Montreal Meetup: Velo Vamos! ML on open bike data}
\item{2015: CapePy Python Meetup Tutorial: Introduction to Machine Learning with Scikit-learn }
\item{2015: BRATS Talk: Standardizing Machine Learning Tasks with Scikit-learn }
\end{rSubsection}
\begin{rSubsection}{Field Trials, Workshops, and Professional Development}{}{}{}
\item{2017-2021 National Canadian Field Robotics Symposium and Field Trials}
\item{2017-2019: Barbados Marine Field Trials}
\item{2017: MILA Deep Learning and Reinforcement Learning Summer School}
\item{2017: McGill Innovation's AI for Social Good Summer Lab, Project on Improving Cycling Transporation in Low-Income Neighborhoods}
\item{2016: IEEE Marine Robotics Summer School}
\item{2016: National Canadian Field Robotics Symposium and Field Trials}
\item{2012-13: SwRI Professional Courses in Proposal Writing, Promoting Research and Development, Technical Writing, \& Project Management}
\item{NAUI Master Scuba Diver, Diving for Science Certified}
\end{rSubsection}
\vspace{-.2cm}
\begin{rSubsection}{Leadership and Volunteer Work}{}{}{}
\item{Reviewer at numerous conferences and workshops including CORL, RSS, ICRA, IROS, and NeurIPS}
\item{2020: Co-organizer and Mentorship Chair of the NeurIPS workshop on Differentiable Vision, Graphics, and Physics (DiffCVGP)}
\item{2020: Co-organizer and Sensors/Sampling Chair of the NeurIPS workshop on AI for Earth Science}
\item{2020: Co-organizer and Sensing/Theory Chair of the ICLR workshop on AI for Earth Science}
\item{2019: Co-organizer of the IROS workshop on Informed Scientific Sampling}
\item{2018: NIPS WiML Volunteer}
\item{2017: ICML Volunteer}
\item{2015: Scikit-learn developer sprint in Paris}
\item{2015: Neural Information Processing Systems (NIPS), Volunteer}
\item{2015: Founder and Technical Organizer of WHOI-Software Technical Group}
\item{2015: CapePy Python Meetup Leader and Member}
\item{2014: Big-data, Robotics, Autonomy, Technology and Sensing (BRATS) Member}
\item{2013: South-Central CleanTech Open Incubator Judge, San Antonio and Austin TX}
\end{rSubsection}
\vspace{-.2cm}
\begin{rSubsection}{Selected Awards}{}{}{}
\item{2019: NCRN Travel Grant}
\item{2017: WiML NIPS Travel Grant} 
\item{2016: McGill GREAT Travel Award} 
\item{2012: UTSA M.S. COE Valero Research Fellowship (declined)}
\item{2013: SwRI Internal Research and Development Funding, Primary Investigator} 
\item{2007: Terry Foundation Scholarship (Complete Undergraduate Funding)} 
\item{2007: Dick Walrath Foundation Scholarship} 
\item{2007: American Quarter Horse Association Scholarship} 
\end{rSubsection}
\vspace{-.2cm}
\begin{rSubsection}{Teaching and Mentorship}{}{}{}
\item{2020: Mentor for undergraduate Mechanical Engineering capstone project developing a sensor recovery system for an autonomous surface vehicle}
\item{2020: Mentor for summer undergraduate work on Robot Manipulator Simulation}
\item{2020: AI4Good Summer Lab Mentor and Advisor for Recycling Sorting Project}
\item{2019: Mentor for undergraduate Mechanical Engineering capstone project developing a sensor deployment system for an autonomous surface vehicle}
\item{2019: AI4Good Summer Lab Mentor and Advisor for Pain Relief Project (winning team)}
\item{2018: AI4Good Summer Lab Mentor and Advisor for AI4Good Project}
\item{2011: Teaching Assistant: EE Signals and Systems}
\item{2011: Teaching Assistant: EE Electronics}
\item{2010: Lab Assistant: EE Microprocessors}
\item{2010: Teaching Assistant: EE Engineering Management}
\end{rSubsection}
\vspace{-.2cm}
\end{rSection}
\begin{rSection}{Oceanographic Research Cruises}
\begin{rSubsection}{Studies of Evolution and Ecology of Petroleum Systems, Gulf of Mexico}{Jun 2015}{R/V Atlantis, Chief Scientist: Dr. David Valentine}{}
\item{Primary software/data processing engineer for Sentry AUV working with multibeam, sidescan, and sub-bottom pipeline data.}
\end{rSubsection}
\begin{rSubsection}{Mapping, Exploration, and Sampling at Havre Volcano, Southwestern Pacific}{Mar 2015}{R/V Revelle, Chief Scientist: Dr. Adam Soule}{}
\item{Primary software/data processing engineer for Sentry AUV in collaboration with Jason ROV. Developed sidescan and sub-bottom pipeline for processing sonar signal using MB-System.}
\end{rSubsection}
\begin{rSubsection}{Monitoring Recovery of Pacific Seamounts, Hawaiian Islands}{Oct 2014}{R/V Sikuliaq, Chief Scientists: Dr. Amy Baco-Taylor and Dr. Brendon Roark}{}
\item{Primary software/data engineer processing subsea navigation and images. Developed classifier for seafloor images for easier processing.}
\end{rSubsection}
\begin{rSubsection}{Juan de Fuca Ridge, Northeastern Pacific}{Jul 2014}{R/V Atlantis, Chief Scientists: Dr. James Kinsey and Dr. Maurice Tivey}{}
\item{Lead software engineer for AUV optical communication system integration. Developed acoustic/optical search algorithm for finding an optical modem on the seafloor. Also provided navigation/data processing and visualization for science.}
\end{rSubsection}
\begin{rSubsection}{Iron Eaters of the Loihi Seamount, Hawaiian Islands}{Jun 2014}{R/V Falkor, Chief Scientist: Dr. Brian Glazer}{}
\item{Primary software/data engineer working with subsea navigation, scientific sensors, and images. Developed thematic map of iron location in images for easy inspection and planning. }
\end{rSubsection}
\begin{rSubsection}{Deep Water Supercoral in Low pH Environments, Gulf of Mexico}{Apr 2014}{R/V Atlantis, Chief Scientist: Dr. Erik Cordes}{}
\item{Primary software/data engineer working with subsea navigation, scientific sensors, and images.}
\end{rSubsection}
%\begin{rSubsection}{AUV Engineering, Buzzards Bay}{Feb 2014}{R/V Tioga}{Dr. Carl Kaiser and Dr. Dana Yoerger}
%\item{Performed engineering testing of Sentry AUV mission executive and control system with several hardware components.}
%\end{rSubsection}
\begin{rSubsection}{Mineral Exploration, Southeastern Pacific}{Jan 2014}{R/V Ka`imikai-O-Kanoloa, Chief Scientist: Dr. Carl Kaiser}{}
\item{Learned AUV deployment, mission planning, data processing, and networking. Developed new initiative for robust data management.}
\end{rSubsection}
\end{rSection}


%\begin{tabular}{ @{} >{\bfseries}l @{\hspace{6ex}} l }
%\end{tabular}
%\end{rSection}


%
%%----------------------------------------------------------------------------------------
%%	TECHNICAL STRENGTHS SECTION
%%----------------------------------------------------------------------------------------
%
%%----------------------------------------------------------------------------------------
%%	EXAMPLE SECTION
%%----------------------------------------------------------------------------------------
%
%%\begin{rSection}{Section Name}
%
%%Section content\ldots
%
%%\end{rSection}
%
%%----------------------------------------------------------------------------------------

\end{document}
