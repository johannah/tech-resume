%%%%%%%%%%%%%%%%%%%%%%%%%%%%%%%%%%%%%%%%%
% Medium Length Professional CV
% LaTeX Template
% Version 2.0 (8/5/13)
%
% This template has been downloaded from:
% http://www.LaTeXTemplates.com
%
% Original author:
% Trey Hunner (http://www.treyhunner.com/)
%
% Important note:
% This template requires the resume.cls file to be in the same directory as the
% .tex file. The resume.cls file provides the resume style used for structuring the
% document.
%
%%%%%%%%%%%%%%%%%%%%%%%%%%%%%%%%%%%%%%%%%

%----------------------------------------------------------------------------------------
%	PACKAGES AND OTHER DOCUMENT CONFIGURATIONS
%----------------------------------------------------------------------------------------

\documentclass{resume} % Use the custom resume.cls style

\usepackage[left=0.75in,top=0.6in,right=0.75in,bottom=0.6in]{geometry} % Document margins

\name{Johanna Hansen} % Your name
%\address{2945 Edouard Montpetit, Apt. 14 \\ Montreal, Quebec H3T1J8} % Your address
\address{ johanna.hansen@mail.mcgill.ca \\ http://johannah.github.io } % Your phone number and email

\begin{document}

%----------------------------------------------------------------------------------------
%	EDUCATION SECTION
%----------------------------------------------------------------------------------------

\begin{rSection}{Education}
{\bf McGill University, Montreal, QC} \hfill {\em 2016 -- 2021} \\ 
\setlength{\parindent}{3ex}
 \indent{Ph.D. in Computer Science in the \emph{Mobile Robotics Lab}} \\
 \indent{Thesis work on \emph{Model-Based Decision Making in Robotic Systems}}\\
  \indent{under the supervision of Dr. Gregory Dudek and Dr. Joelle Pineau}\\
\noindent {\bf University of Texas at San Antonio, San Antonio, TX} \hfill {\em 2012 -- 2015} \\ 
\indent {Graduate coursework (30 hours) in Electrical Engineering, \emph{Digital Signal Processing}} \\
\noindent {{\bf Texas State University, San Marcos, TX} \hfill {\em 2007 -- 2011}} \\
\indent{B.S. in Electrical Engineering, \emph{Networking and Communication} } \\
\indent{Capstone Project on \emph{Identifying Devices in the Power Grid by their Current Signatures}} \\
\indent{B.S. Resource and Environmental Geography}

\noindent \textbf{Selected Coursework}: Machine Learning, Reinforcement Learning, Robotics, Digital Signal Processing, Numerical Estimation, Communication Systems, Networking, Data Compression and Error Coding,  Geographic Information Systems, Physical Geography, Environmental Resource Management

\end{rSection}

\begin{rSection}{Academic Papers}

\textbf{Hansen, J.}, Manjanna, S.,  Quattrini, L. A., Rekleitis, I., Dudek, G., \textit{Autonomous Marine Sampling Enhanced by Strategically Deployed Drifters in Marine Flow Fields}, MTS/IEEE OCEANS, 2018, \emph{{(Top 20 Student Submission)}}.

\textbf{Hansen, J.}, Dudek, G., \textit{Coverage Optimization with Non-Actuated, Floating Mobile Sensors using Iterative Trajectory Planning in Marine Flow Fields}, IEEE International Conference on Intelligent Robots (IROS), 2018.

\textbf{Hansen, J.}*, Kastner, K.*, Courville, A., Dudek, G.,  \textit{Planning in Dynamic Environments with Conditional Autoregressive Models}, International Conference on Machine Learning (ICML), workshop on Prediction and Generative Modeling in Reinforcement Learning, 2018.

Henderson P., Chang, W.D., Shkurti, F., \textbf{Hansen, J.}, Meger, D., Dudek G., 
\textit{Benchmark Environments for Multitask Learning in Continuous Domains}, International Conference on Machine Learning (ICML), workshop on Lifelong Learning, 2018, https://arxiv.org/abs/1708.04352. 

Manjanna, S., \textbf{ Hansen, J. }, Quattrini, L. A., Rekleitis, I., Dudek, G., 
\textit{Collaborative Sampling Using Heterogeneous Marine Robots Driven by Visual Cues}, Canadian Conference on Computer and Robot Vision (CRV), 2017. 
                
Quattrini L. A., Rekleitis, I., Manjanna, S., Kakodkar, N., \textbf{Hansen, J.},   Dudek, G.,  Bobadilla, L.,  Anderson, J., and Smith, R.,
            \textit{Data Correlation and Comparison from Multiple Sensors over a Coral Reef with a Team of Heterogeneous Aquatic Robots},
             International Symposium on Experimental Robotics (ISER),
              2016.

\textbf{Hansen, J.}, Fourie, D., Kinsey, J., Pontbriand, C., Ware, J., Farr, N., Kaiser, C., and Tivey, M., \textit{Autonomous Acoustic-Aided Optical Localization for Data Transfer}, MTS/IEEE OCEANS, 2015.

Pontbriand, C., Farr, N., Fourie, D., \textbf{Hansen, J.},  Kinsey, J., Pelletier, J., and Ware, J., 
 \textit{Wireless Data Harvesting Using the AUV Sentry and WHOI Optical Modem}, MTS/IEEE OCEANS, 2015.

\textbf{Hansen, J.}, Wilden, G., Abbott, B., and Green, R., \textit{The Ultrasonic Culvert
Inspection System (UCIS): A Low-Cost Device for Conduit Inspection}, 2014 Transportation
Research Board 93rd Annual Meeting. 
\end{rSection}
\newpage





%----------------------------------------------------------------------------------------
%	WORK EXPERIENCE SECTION
%----------------------------------------------------------------------------------------

\begin{rSection}{Experience}

\begin{rSubsection}{NASA Jet Propulsion Lab (JPL)}{June 2019--Sept 2019}{Research Intern, Mobility and Robotics Section}{Pasadena, CA}
\item{Developed a lightweight model which learned to switch between terrain-relative and direct localization experts to facilitate autonomous rover sample tube collection for the Mars Sample Return Project.}
\item{Built a specialized dataset resembling Mars deployment conditions for evaluating specific algorithmic failures}
\end{rSubsection}

\begin{rSubsection}{McGill University}{Jan 2016--current}{Graduate Researcher, Mobile Robotics Lab}{Montreal, QC}
\item{Research on model-based planning and reinforcement learning with generative models for mobile robots}
\item{Collaborated with ecologists to facilitate automatic classification and modeling of zooplankton in Canadian lakes}
\item{Research on strategic marine sampling with custom low-cost floating sensors and autonomous boats}
\item{Built system for portable underwater localization using low-cost USBL/GPS components} 
\item{Developed technique for calculating total iceberg volume using images captured above the  waterline and sonar reflections below water}
\end{rSubsection}

\begin{rSubsection}{Woods Hole Oceanographic Institution (WHOI)}{Jan 2014 -- Sept 2015}{Autonomous Underwater Vehicle Engineer, National Deep Submergence Facility}{Woods Hole, MA}
\item{Software/Data/Electrical Engineer for deep-diving autonomous underwater vehicles (AUVs) working in research and ship-board operational environments.}
\item{Assisted in all aspects of at-sea operations including deployments, hardware repair, and dive planning.}
\item{Primary data scientist at-sea for geophysical, acoustic, and image processing.}
\item{Developed automated spatio-temporal processing pipeline for high-resolution multibeam, sidescan sonar, and optical data maps.}
\item{Assisted in overhaul of navigation (GPS/USBL/LBL/DR) processing and visualization.}
\item{Developed user-interfaces (QT), vehicle control code (C++), data processing code (Python/Matlab), and device drivers primarily in Linux.}
\end{rSubsection}
%%------------------------------------------------
%
\begin{rSubsection}{Southwest Research Institute (SwRI)}{Jan 2012 -- Dec 2013}{Engineer, Automation and Data Systems Division}{San Antonio, TX}{}
\item{Software/Electrical Applied Research Engineer for research, commercial, and government clients.}
\item{Primary end-to-end software engineer building a mapping sensor consisting of acoustic transducers, DSP, camera, and embedded computer with remote control and interpretation.}
\item{Developed sampling, filtering, visualization scheme for live acoustic data.}
\item{Wrote beamforming calibration routine to tune for errors in sensor fabrication.}
\end{rSubsection}


\begin{rSubsection}{Lower Colorado River Authority (LCRA)}{Jan 2011 -- Dec 2011}{Engineering Coop, Telecommunications Department}{Austin, TX}
\item{Designed and configured SONET, optical fiber, Ethernet, and microwave systems for critical communication infrastructure including power generation/distribution, dam and 
            irrigation control, and emergency response coordination. }
\item{Project lead for pilot irrigation control system using networked 900 MHz Radios.}
\end{rSubsection}
\end{rSection}
\newpage
\begin{rSection}{Oceanographic Research Cruises}

\begin{rSubsection}{Studies of Evolution and Ecology of Petroleum Systems, Gulf of Mexico}{Jun 2015}{R/V Atlantis, Chief Scientist: Dr. David Valentine}{}
\item{Primary software/data processing engineer for Sentry AUV working with multibeam, sidescan, and sub-bottom pipeline data.}
\end{rSubsection}

\begin{rSubsection}{Mapping, Exploration, and Sampling at Havre Volcano, Southwestern Pacific}{Mar 2015}{R/V Revelle, Chief Scientist: Dr. Adam Soule}{}
\item{Primary software/data processing engineer for Sentry AUV in collaboration with Jason ROV. Developed sidescan and sub-bottom pipeline for processing sonar signal using MB-System.}
\end{rSubsection}

\begin{rSubsection}{Monitoring Recovery of Pacific Seamounts, Hawaiian Islands}{Oct 2014}{R/V Sikuliaq, Chief Scientists: Dr. Amy Baco-Taylor and Dr. Brendon Roark}{}
\item{Primary software/data engineer processing subsea navigation and images. Developed classifier for seafloor images for easier processing.}
\end{rSubsection}

\begin{rSubsection}{Juan de Fuca Ridge, Northeastern Pacific}{Jul 2014}{R/V Atlantis, Chief Scientists: Dr. James Kinsey and Dr. Maurice Tivey}{}
\item{Lead software engineer for AUV optical communication system integration. Developed acoustic/optical search algorithm for finding an optical modem on the seafloor. Also provided navigation/data processing and visualization for science.}
\end{rSubsection}

\begin{rSubsection}{Iron Eaters of the Loihi Seamount, Hawaiian Islands}{Jun 2014}{R/V Falkor, Chief Scientist: Dr. Brian Glazer}{}
\item{Primary software/data engineer working with subsea navigation, scientific sensors, and images. Developed thematic map of iron location in images for easy inspection and planning. }
\end{rSubsection}

\begin{rSubsection}{Deep Water Supercoral in Low pH Environments, Gulf of Mexico}{Apr 2014}{R/V Atlantis, Chief Scientist: Dr. Erik Cordes}{}
\item{Primary software/data engineer working with subsea navigation, scientific sensors, and images.}
\end{rSubsection}

%\begin{rSubsection}{AUV Engineering, Buzzards Bay}{Feb 2014}{R/V Tioga}{Dr. Carl Kaiser and Dr. Dana Yoerger}
%\item{Performed engineering testing of Sentry AUV mission executive and control system with several hardware components.}
%\end{rSubsection}

\begin{rSubsection}{Mineral Exploration, Southeastern Pacific}{Jan 2014}{R/V Ka`imikai-O-Kanoloa, Chief Scientist: Dr. Carl Kaiser}{}
\item{Learned AUV deployment, mission planning, data processing, and networking. Developed new initiative for robust data management.}
\end{rSubsection}

\end{rSection}

\begin{rSection}{Professional Activities}
\begin{rSubsection}{Technical Skills}{}{}{}
\item{\emph{Software}: scientific and machine learning Python (PyTorch, NumPy, scikit-learn, Pandas, OpenCV and more), C/C++, Matlab, Bash, ROS, GIS on Linux, OSX, Windows}
\item{\emph{Hardware}: embedded systems, localization systems, marine instrumentation and sensing}
\item{\emph{Data}: geospatial and time series data, image processing, sonar, navigation }
\item{\emph{Languages}: English (native), Spanish (basic), French (learning)}
\end{rSubsection}

\begin{rSubsection}{Presentations}{}{}{}
\item{2019: Tutorial on Model-Based Reinforcement Learning at AI4Good Workshop}
\item{2019: GRIL Invited Workshop Presentation on Robotic Sampling in Aquatic Environments}
\item{2017: NIPS WiML Workshop Poster Presentation on Distributed Sensors}
\item{2015: CapePy Python Meetup Tutorial: Introduction to Machine Learning with Scikit-learn }
\item{2015: SciPy 2015 Talk: Characterizing the Seafloor with Python as a Toolbox}
\item{2015: BRATS Talk: Standardizing Machine Learning Tasks with Scikit-learn }
\end{rSubsection}

\begin{rSubsection}{Workshops and Professional Development}{}{}{}
\item{2016-2019: Barbados Marine Field Trials}
\item{2017: MILA Deep Learning and Reinforcement Learning Summer School}
\item{2017: McGill Innovation's AI for Social Good Summer Lab}
\item{2017: National Canadian Field Robotics Symposium and Field Trials}
\item{2016: IEEE Marine Robotics Summer School}
\item{2016: National Canadian Field Robotics Symposium and Field Trials}
\item{2013: SciPy: Scikit-learn, Cython, Geospatial Tutorials and Tracks}
\item{2013: SwRI Professional Course in Proposal Writing}
\item{2013: SwRI Professional Course in Promoting Research and Development}
\item{2012: SwRI Professional Course in Technical Writing}
\item{2012: SwRI Professional Course in Project Management}
\item{2010: NAUI Master Scuba Diver, Diving for Science Certified}
\end{rSubsection}


\begin{rSubsection}{Leadership and Volunteer Work}{}{}{}
\item{2018: NIPS WiML Volunteer}
\item{2017: ICML Volunteer}
\item{2015: Scikit-learn developer sprint in Paris}
\item{2015: Neural Information Processing Systems (NIPS), Volunteer}
\item{2015: Founder and Technical Organizer of WHOI-Software Technical Group}
\item{2015: CapePy Python Meetup Leader and Member}
\item{2014: Big-data, Robotics, Autonomy, Technology and Sensing (BRATS) Member}
\item{2013: South-Central CleanTech Open Incubator Judge, San Antonio and Austin TX}
\end{rSubsection}

\begin{rSubsection}{Selected Awards}{}{}{}
\item{2017: WiML NIPS Travel Grant} 
\item{2016: McGill GREAT Travel Award} 
\item{2012: UTSA M.S. COE Valero Research Fellowship Offer}
\item{2013: Internal Research and Development Funding, Primary Investigator} 
\item{2007: Terry Foundation Scholarship (Complete Undergraduate Funding)} 
\item{2007: Dick Walrath Foundation Scholarship} 
\item{2007: American Quarter Horse Association Scholarship} 
\end{rSubsection}
\end{rSection}
\begin{rSection}{Teaching Experience}
\begin{rSubsection}{Texas State University}{Jan 2010--Dec 2010}{}{}
\item{Signals and Systems Teaching Assistant}
\item{Electronics Teaching Assistant}
\item{Microprocessors Lab Assistant}
\item{Engineering Management Teaching Assistant}
\end{rSubsection}
\end{rSection}




%\begin{tabular}{ @{} >{\bfseries}l @{\hspace{6ex}} l }
%\end{tabular}
%\end{rSection}


%
%%----------------------------------------------------------------------------------------
%%	TECHNICAL STRENGTHS SECTION
%%----------------------------------------------------------------------------------------
%
%%----------------------------------------------------------------------------------------
%%	EXAMPLE SECTION
%%----------------------------------------------------------------------------------------
%
%%\begin{rSection}{Section Name}
%
%%Section content\ldots
%
%%\end{rSection}
%
%%----------------------------------------------------------------------------------------

\end{document}
