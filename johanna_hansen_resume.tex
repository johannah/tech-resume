%%%%%%%%%%%%%%%%%%%%%%%%%%%%%%%%%%%%%%%%%
% Medium Length Professional CV
% LaTeX Template
% Version 2.0 (8/5/13)
%
% This template has been downloaded from:
% http://www.LaTeXTemplates.com
%
% Original author:
% Trey Hunner (http://www.treyhunner.com/)
%
% Important note:
% This template requires the resume.cls file to be in the same directory as the
% .tex file. The resume.cls file provides the resume style used for structuring the
% document.
%
%%%%%%%%%%%%%%%%%%%%%%%%%%%%%%%%%%%%%%%%%

%----------------------------------------------------------------------------------------
%	PACKAGES AND OTHER DOCUMENT CONFIGURATIONS
%----------------------------------------------------------------------------------------

\documentclass{resume} % Use the custom resume.cls style

\usepackage[left=0.75in,top=0.6in,right=0.75in,bottom=0.6in]{geometry} % Document margins

\name{Johanna Hansen} % Your name
%\address{Montreal, Quebec} % Your address
\address{ johanna.hansen@mail.mcgill.ca \\ http://johannah.github.io } % Your phone number and email
\begin{document}
\vspace{-.5cm}
\begin{rSection}{Education}
{\bf McGill University, Montreal, QC} \hfill {\em 2016 -- 2022} \\ 
\setlength{\parindent}{3ex}
\indent{Ph.D. in Computer Science (expected), Mobile Robotics Lab} \\
\indent{Learning Robotic Policies with Physically Consistent World Models}\\
%\indent{under the supervision of Dr. Gregory Dudek and Dr. Joelle Pineau}\\
\noindent {\bf University of Texas at San Antonio, San Antonio, TX} \hfill {\em 2012 -- 2015} \\ 
\indent {Graduate coursework (30 hours) in Electrical Engineering, Digital Signal Processing} \\
\noindent {{\bf Texas State University, San Marcos, TX} \hfill {\em 2007 -- 2011}} \\
\indent{B.S. in Electrical Engineering and B.S. in Environmental Geography} \\
%\indent{Capstone Project on Identifying Devices in the Power Grid by their Current Signatures} \\
	\vspace{-1cm}
\end{rSection}
\begin{rSection}{Technical Skills}{}{}{}
\noindent{\textbf{Expertise:} Robotics, Machine/Reinforcement Learning, Perception, Sensing, Environmental Science } \\
\noindent{\textbf{Software}: Scientific Python, Physics Simulation, ROS, C, Matlab} \\
	\vspace{-1cm}
\end{rSection}
\begin{rSection}{Experience}
\begin{rSubsection}{McGill University}{Jan 2016--current}{Graduate Researcher, Mobile Robotics Lab / Mila}{Montreal, QC}
\item{Model-based planning and reinforcement learning with physics-grounded, learned world models. }
\end{rSubsection}
\vspace{-.2cm}
	\begin{rSubsection}{Samsung AI Center (SAIC)}{Jan 2021-current}{Part-Time Research Intern, Tactile Sensing Group}{Montreal, QC}
\item{Pixel-based multitask learning with visuotactile-based grounding for complex manipulation tasks. }
\end{rSubsection}
\vspace{-.2cm}
\begin{rSubsection}{NASA Jet Propulsion Lab (JPL)}{Summer 2019}{Research Intern and Remote Affiliate, Mobility and Robotics Section}{Pasadena, CA}
\item{Developed SOTA geometric and direct object localization methods for Mars Sample Return Mission. }
\end{rSubsection}
	\vspace{-.2cm}
\begin{rSubsection}{Woods Hole Oceanographic Institution (WHOI)}{Jan 2014 -- Sept 2015}{Autonomous Underwater Vehicle Engineer, National Deep Submergence Facility}{Woods Hole, MA}
\item{Software/Data/Electrical Engineer for deep-diving autonomous underwater vehicles (AUVs) working in research and ship-board operational environments in scientific instrumentation and visualization.} 
\end{rSubsection}
	\vspace{-.2cm}
\begin{rSubsection}{Southwest Research Institute (SwRI)}{Jan 2012 -- Dec 2013}{Engineer, Automation and Data Systems Division}{San Antonio, TX}{}
%\item{Software/Electrical Applied Research Engineer for research, commercial, and government clients.}
\item{Primary software engineer building a new live acoustic/visual mapping sensor for inspecting conduits. }
\end{rSubsection}
\end{rSection}
	\vspace{-.5cm}
 \begin{rSection}{Selected Academic Papers}

\textbf{Hansen, J.*}, Kastner, K.*, Huang, Y., Courville, A., Meger, D., Dudek, G., \textit{Learning to Manipulate from Pixels on Rigid Body Robots with a Kinematic Critic}, (under review), 2022

\textbf{Hansen, J.}, Hogan, F., Rivkin, D., Meger, D., Jenkin, M., Dudek, G., \textit{Visuotactile-RL: Learning Multimodal Manipulation Policies with Deep Reinforcement Learning}, ICRA, 2022


\textbf{Hansen, J.}, Manjanna, S.,  Quattrini, L. A., Rekleitis, I., Dudek, G., \textit{Autonomous Marine Sampling Enhanced by Strategically Deployed Drifters}, IEEE OCEANS, 2018, \emph{{(Top 20 Student Submission)}}.

\textbf{Hansen, J.}, Dudek, G., \textit{Coverage Optimization with Non-Actuated, Floating Mobile Sensors using Iterative Trajectory Planning in Marine Flow Fields}, IEEE International Conference on Intelligent Robots (IROS), 2018.

\textbf{Hansen, J.}*, Kastner, K.*, Courville, A., Dudek, G.,  \textit{Planning in Dynamic Environments with Conditional Autoregressive Models}, International Conference on Machine Learning (ICML), workshop on Prediction and Generative Modeling in Reinforcement Learning, 2018.

% \end{rSection}
% \vspace{-.2cm}
% \begin{rSection}{Professional Activities}
% \begin{rSubsection}{Invited Presentations}{}{}{}
% \item{2019: Tutorial on Model-Based Reinforcement Learning at AI4Good Workshop}
% \item{2019: GRIL Presentation on Robotic Sampling in Aquatic Environments}	
% \item{2018: PyLadies Montreal Meetup: Velo Vamos! ML on open bike data}
% \item{2015: CapePy Python Meetup Tutorial: Introduction to Machine Learning with Scikit-learn }
% \item{2015: BRATS Talk: Standardizing Machine Learning Tasks with Scikit-learn }
% \end{rSubsection}
% \vspace{-.2cm}
% \begin{rSubsection}{Field Trials, Workshops, and Professional Development}{}{}{}
% \item{2017-2019: Barbados Marine Field Trials}
% \item{2017-2019 National Canadian Field Robotics Symposium and Field Trials}
% \item{2017: MILA Deep Learning and Reinforcement Learning Summer School}
% \item{2017: McGill Innovation's AI for Social Good Summer Lab}
% \item{2016: IEEE Marine Robotics Summer School}
% \item{2016: National Canadian Field Robotics Symposium and Field Trials}
% \item{2012-13: SwRI Professional Courses in Proposal Writing, Promoting Research and Development, Technical Writing, \& Project Management}
% \item{NAUI Master Scuba Diver, Diving for Science Certified}
% \end{rSubsection}
% %\begin{rSubsection}{Leadership and Volunteer Work}{}{}{}
% %\item{2018: NIPS WiML Volunteer}
% %\item{2017: ICML Volunteer}
% %\item{2015: Scikit-learn developer sprint in Paris}
% %\item{2015: Neural Information Processing Systems (NIPS), Volunteer}
% %\item{2015: Founder and Technical Organizer of WHOI-Software Technical Group}
% %\item{2015: CapePy Python Meetup Leader and Member}
% %\item{2014: Big-data, Robotics, Autonomy, Technology and Sensing (BRATS) Member}
% %\item{2013: South-Central CleanTech Open Incubator Judge, San Antonio and Austin TX}
% %\end{rSubsection}
% \vspace{-.2cm}
% \begin{rSubsection}{Selected Awards}{}{}{}
% \item{2019: NCRN Travel Grant}
% \item{2017: WiML NIPS Travel Grant} 
% \item{2016: McGill GREAT Travel Award} 
% \item{2012: UTSA M.S. COE Valero Research Fellowship Offer}
% \item{2013: Internal Research and Development Funding, Primary Investigator} 
% \item{2007: Terry Foundation Scholarship (Complete Undergraduate Funding)} 
% \item{2007: Dick Walrath Foundation Scholarship} 
% \item{2007: American Quarter Horse Association Scholarship} 
% \end{rSubsection}
% \vspace{-.2cm}
% \end{rSection}
% \begin{rSection}{Teaching Experience}
% \begin{rSubsection}{Texas State University}{Jan 2010--Dec 2010}{}{}
% \item{Signals and Systems Teaching Assistant}
% \item{Electronics Teaching Assistant}
% \item{Microprocessors Lab Assistant}
% \item{Engineering Management Teaching Assistant}
% \end{rSubsection}
% \end{rSection}
% \vspace{-.2cm}
% \begin{rSection}{Oceanographic Research Cruises}
% \begin{rSubsection}{Studies of Evolution and Ecology of Petroleum Systems, Gulf of Mexico}{Jun 2015}{R/V Atlantis, Chief Scientist: Dr. David Valentine}{}
% \item{Primary software/data processing engineer for Sentry AUV working with multibeam, sidescan, and sub-bottom pipeline data.}
% \end{rSubsection}
% \begin{rSubsection}{Mapping, Exploration, and Sampling at Havre Volcano, Southwestern Pacific}{Mar 2015}{R/V Revelle, Chief Scientist: Dr. Adam Soule}{}
% \item{Primary software/data processing engineer for Sentry AUV in collaboration with Jason ROV. Developed sidescan and sub-bottom pipeline for processing sonar signal using MB-System.}
% \end{rSubsection}
% \begin{rSubsection}{Monitoring Recovery of Pacific Seamounts, Hawaiian Islands}{Oct 2014}{R/V Sikuliaq, Chief Scientists: Dr. Amy Baco-Taylor and Dr. Brendon Roark}{}
% \item{Primary software/data engineer processing subsea navigation and images. Developed classifier for seafloor images for easier processing.}
% \end{rSubsection}
% \begin{rSubsection}{Juan de Fuca Ridge, Northeastern Pacific}{Jul 2014}{R/V Atlantis, Chief Scientists: Dr. James Kinsey and Dr. Maurice Tivey}{}
% \item{Lead software engineer for AUV optical communication system integration. Developed acoustic/optical search algorithm for finding an optical modem on the seafloor. Also provided navigation/data processing and visualization for science.}
% \end{rSubsection}
% \begin{rSubsection}{Iron Eaters of the Loihi Seamount, Hawaiian Islands}{Jun 2014}{R/V Falkor, Chief Scientist: Dr. Brian Glazer}{}
% \item{Primary software/data engineer working with subsea navigation, scientific sensors, and images. Developed thematic map of iron location in images for easy inspection and planning. }
% \end{rSubsection}
% \begin{rSubsection}{Deep Water Supercoral in Low pH Environments, Gulf of Mexico}{Apr 2014}{R/V Atlantis, Chief Scientist: Dr. Erik Cordes}{}
% \item{Primary software/data engineer working with subsea navigation, scientific sensors, and images.}
% \end{rSubsection}
% %\begin{rSubsection}{AUV Engineering, Buzzards Bay}{Feb 2014}{R/V Tioga}{Dr. Carl Kaiser and Dr. Dana Yoerger}
% %\item{Performed engineering testing of Sentry AUV mission executive and control system with several hardware components.}
% %\end{rSubsection}
% \begin{rSubsection}{Mineral Exploration, Southeastern Pacific}{Jan 2014}{R/V Ka`imikai-O-Kanoloa, Chief Scientist: Dr. Carl Kaiser}{}
% \item{Learned AUV deployment, mission planning, data processing, and networking. Developed new initiative for robust data management.}
% \end{rSubsection}
% \end{rSection}


%\begin{tabular}{ @{} >{\bfseries}l @{\hspace{6ex}} l }
%\end{tabular}
%\end{rSection}


%
%%----------------------------------------------------------------------------------------
%%	TECHNICAL STRENGTHS SECTION
%%----------------------------------------------------------------------------------------
%
%%----------------------------------------------------------------------------------------
%%	EXAMPLE SECTION
%%----------------------------------------------------------------------------------------
%
%%\begin{rSection}{Section Name}
%
%%Section content\ldots
%
%%\end{rSection}
%
%%----------------------------------------------------------------------------------------

\end{document}
